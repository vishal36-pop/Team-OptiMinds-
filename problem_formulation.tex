\documentclass[11pt, a4paper]{article}
\usepackage[utf8]{inputenc}
\usepackage[margin=1in]{geometry}
\usepackage{amsmath}
\usepackage{amssymb}
\usepackage{amsthm}
\usepackage{booktabs}
\usepackage{hyperref}
\usepackage{fancyhdr}
\usepackage{float}

% Header/Footer Setup
\pagestyle{fancy}
\fancyhf{}
\lhead{OptiMinds: Project Formulation}
\rhead{Optimization Methods}
\cfoot{\thepage}

\title{\textbf{Problem Formulation: \\ Robust Economic Dispatch with Uncertain Demand}}
\author{Team OptiMinds \\ \small{Abhinav Reddy Alwala, Lohith Pasumarthi, Vishal Reddy Kondakindi}}
\date{\today}

\begin{document}

\maketitle

\section{Problem Definition}

We study the Economic Dispatch problem under uncertain demand, where a set of 
$n$ parallel machines must be allocated production loads in a cost-efficient and 
reliable manner. Each machine exhibits a convex quadratic operating cost and is 
restricted by both minimum and maximum capacity limits. The goal is to determine 
how much each machine should produce so that the total operating cost is 
minimized while ensuring that demand is met with a high level of reliability.

Unlike classical formulations that assume fixed and perfectly known demand, our 
model explicitly incorporates randomness through a probabilistic service-level 
constraint. This leads to a convex yet uncertainty-aware optimization problem 
that captures the trade-off between economic efficiency and reliability.

\subsection{Motivation}

Real-world manufacturing systems rarely operate under deterministic conditions. 
Customer demand fluctuates due to market variability, supply-chain delays, and 
forecasting errors. Relying solely on expected demand can therefore result in 
underproduction, unmet service levels, or excessive reliance on expensive 
emergency production.

This motivates the need for an optimization framework that:
\begin{itemize}
    \item incorporates random demand directly into the formulation,
    \item enforces a target reliability level $1-\alpha$,
    \item respects machine-specific capacity constraints, and
    \item remains convex and computationally tractable.
\end{itemize}

By reformulating chance constraints into deterministic equivalents, our model 
produces reliable and cost-efficient production plans suitable for practical 
industrial settings.

\subsection{Notation}

\begin{table}[H]
    \centering
    \begin{tabular}{l l}
    \toprule
    \textbf{Symbol} & \textbf{Description} \\
    \midrule
    $k \in \{1,\dots,n\}$ & Machine index \\
    $x_k$ & Production quantity allocated to machine $k$ \\
    $a_k, b_k, c_k$ & Cost parameters of machine $k$ ($a_k>0$ ensures convexity) \\
    $\ell_k, u_k$ & Minimum and maximum capacity limits of machine $k$ \\
    $D$ & Random demand with mean $\mu$ and variance $\sigma^2$ \\
    $\alpha$ & Acceptable violation probability (risk level) \\
    \bottomrule
    \end{tabular}
\end{table}
\section{Mathematical Model}

\subsection{Objective Function}
The operating cost of each machine is modeled as a strictly convex quadratic function. We seek to minimize the total system cost:
\begin{equation}
    \min_{\mathbf{x}} f(\mathbf{x}) = \sum_{k=1}^{n} \left( a_k x_k^2 + b_k x_k + c_k \right)
\end{equation}
\subsection{Marginal Cost Analysis}

\paragraph{Definition.}
The \emph{marginal cost} of machine $k$ is defined as the rate at which its total
operating cost increases with respect to its production level. Formally, it is
the first derivative of the cost function with respect to $x_k$:
\[
MC_k(x_k) = \frac{dC_k(x_k)}{dx_k}.
\]
For each machine $k$, the total operating cost is modeled as
\[
C_k(x_k) = \tfrac{1}{2}a_k x_k^2 + b_k x_k + c_k,
\qquad a_k > 0.
\]

The marginal cost is obtained by differentiating the cost function with respect to the production level $x_k$:
\[
\frac{dC_k}{dx_k}
= \frac{d}{dx_k}\left( \tfrac{1}{2}a_k x_k^2 + b_k x_k + c_k \right)
= a_k x_k + b_k.
\]

\paragraph{Interpretation.}
The marginal cost is linear because:
\begin{itemize}
    \item the quadratic term $\tfrac{1}{2}a_k x_k^2$ represents increasing inefficiency or wear as the machine operates at higher output levels;
    \item the derivative of a quadratic term is linear, leading to the component $a_k x_k$;
    \item the term $b_k$ corresponds to constant per-unit operating costs (such as energy or materials);
    \item the fixed cost $c_k$ does not influence marginal cost since its derivative is zero.
\end{itemize}

Thus, the incremental cost of producing one additional unit on machine $k$ is
\[
MC_k(x_k) = a_k x_k + b_k.
\]

\paragraph{Remark on Minimum Capacity.}
In our formulation, the lower capacity bound is typically set to
\[
\ell_k = 0,
\]
which reflects the practical interpretation that a machine may be fully shut down (i.e., produce nothing) when it is not required.


\subsection{Constraints}
The system is subject to physical capacity limits and a probabilistic demand requirement:

\begin{align}
    \textbf{Capacity:} & \quad \ell_k \le x_k \le u_k \quad \forall k = 1, \dots, n \\
    \textbf{Reliability:} & \quad \mathbb{P}\left( \sum_{k=1}^{n} x_k \ge D \right) \ge 1 - \alpha
\end{align}

\section{Deterministic Reformulation}

To solve the stochastic service-level constraint using convex optimization, we
convert the chance constraint into a deterministic inequality:
\[
    \sum_{k=1}^{n} x_k \ge D_{\text{eff}},
\]
where $D_{\text{eff}}$ is the effective demand threshold.

\subsection{Model 1: Gaussian Approximation (Standard)}

Starting from the probabilistic requirement:
\[
    \mathbb{P}(D \le S) \ge 1 - \alpha, 
    \qquad S = \sum_{k=1}^n x_k,
\]
we standardize the demand random variable:
\[
    Z = \frac{D - \mu}{\sigma},
    \qquad 
    t = \frac{S - \mu}{\sigma}.
\]

Thus,
\[
    \mathbb{P}(Z \le t) \ge 1 - \alpha
    \quad\Longleftrightarrow\quad
    \mathbb{P}(Z > t) \le \alpha.
\]

By definition of the upper-tail quantile \(z_\alpha\) of the standard normal
distribution:
\[
    \mathbb{P}(Z > z_\alpha) = \alpha,
\]
we obtain the deterministic requirement:
\[
    t \ge z_\alpha
    \quad \Longrightarrow \quad
    S \ge \mu + z_\alpha \sigma.
\]

Hence the effective demand threshold under Gaussian demand is:
\[
    D_{\text{eff}}^{\text{Normal}}
    = \mu + z_\alpha \sigma.
\]

\subsection{Model 2: Distributionally Robust(no assumption on the distribution of )}

When only the mean and variance are known, without assuming normality, we use
the One-Sided Chebyshev Inequality:
\[
    \mathbb{P}(D - \mu \ge t\sigma) \le \frac{1}{1+t^2}.
\]

Imposing the reliability level:
\[
    \frac{1}{1+t^2} = \alpha
    \quad \Rightarrow \quad
    t = \sqrt{\frac{1-\alpha}{\alpha}}.
\]

Thus the distributionally robust effective demand becomes:
\[
    D_{\text{eff}}^{\text{Robust}}
    = \mu + \sqrt{\frac{1-\alpha}{\alpha}} \, \sigma.
\]

\textit{Model 2 is more conservative, reflecting the Price of Robustness.}


\section{KKT Conditions}
We analyze the optimality conditions for the deterministic problem. First, we convert all constraints into the standard form $g_i(\mathbf{x}) \le 0$.

\subsection{Standard Form Transformation}
\begin{enumerate}
    \item \textbf{Demand Constraint:} $\sum x_k \ge D_{\text{eff}} \implies D_{\text{eff}} - \sum_{k=1}^{n} x_k \le 0$
    \item \textbf{Max Capacity:} $x_k \le u_k \implies x_k - u_k \le 0$
    \item \textbf{Min Capacity:} $x_k \ge \ell_k \implies \ell_k - x_k \le 0$
\end{enumerate}

Let the Lagrange multipliers be $\mu_0$ (demand), $\mu_{u,k}$ (upper bounds), and $\mu_{\ell,k}$ (lower bounds).

\subsection{The Lagrangian}
\begin{equation}
    \mathcal{L}(\mathbf{x}, \boldsymbol{\mu}) = \sum_{k=1}^{n}(a_k x_k^2 + b_k x_k) + \mu_0 \left( D_{\text{eff}} - \sum_{k=1}^{n} x_k \right) + \sum_{k=1}^{n} \mu_{u,k}(x_k - u_k) + \sum_{k=1}^{n} \mu_{\ell,k}(\ell_k - x_k)
\end{equation}

\subsection{Necessary and Sufficient Conditions}
Since the problem is convex, the KKT conditions are necessary and sufficient for the global optimum $\mathbf{x}^*$.

\paragraph{1. Primal Feasibility ($g_i(\mathbf{x}^*) \le 0$)}
\begin{align}
    D_{\text{eff}} - \sum_{k=1}^{n} x_k^* &\le 0 \\
    x_k^* - u_k &\le 0 \quad \forall k \\
    \ell_k - x_k^* &\le 0 \quad \forall k
\end{align}

\paragraph{2. Dual Feasibility ($\mu \ge 0$)}
\begin{equation}
    \mu_0 \ge 0, \quad \mu_{u,k} \ge 0, \quad \mu_{\ell,k} \ge 0 \quad \forall k
\end{equation}

\paragraph{3. Stationarity ($\nabla_\mathbf{x} \mathcal{L} = 0$)}
For each machine $k$, the gradient vanishes:
\begin{equation}
    (2a_k x_k^* + b_k) - \mu_0 + \mu_{u,k} - \mu_{\ell,k} = 0
\end{equation}
\textit{Interpretation: For unconstrained machines ($\mu_{u,k}=\mu_{\ell,k}=0$), the marginal cost $2a_k x_k^* + b_k$ must equal the system shadow price $\mu_0$.}

\paragraph{4. Complementary Slackness ($\mu_i g_i(\mathbf{x}^*) = 0$)}
\begin{align}
    \mu_0 \left( D_{\text{eff}} - \sum_{k=1}^{n} x_k^* \right) &= 0 \\
    \mu_{u,k} (x_k^* - u_k) &= 0 \quad \forall k \\
    \mu_{\ell,k} (\ell_k - x_k^*) &= 0 \quad \forall k
\end{align}

\end{document}g